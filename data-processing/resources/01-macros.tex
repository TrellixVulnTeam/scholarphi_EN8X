% Convenience commands for checking if command is defined or not. Work in TeX and LaTeX
% If possible, check for definitions using e-TeX's '\ifcsname' command, as it does not
% have the side effect of defining an undefined command as '\relax'. See
% the discussion about this side effect here https://tex.stackexchange.com/a/164197/198728.
% Assume by default that there is a macro named '\ifcsname'
\def\scholarifdefinedelse#1#2#3{%
\ifcsname#1\endcsname%
#2%
\else%
#3%
\fi%
}

% If there is not, then redefine the definition macro to use lower-level macros.
\expandafter\ifx\csname ifcsname\endcsname\relax%
\def\scholarifdefinedelse#1#2#3{%
\expandafter\ifx\csname #1\endcsname\relax%
#3%
\else%
#2%
\fi%
}
\else
\fi

\def\scholarifundefined#1#2{%
\scholarifdefinedelse{#1}{}{#2}%
}
\def\scholarifdefined#1#2{%
\scholarifdefinedelse{#1}{#2}{}%
}


% Define names of color drivers
% Once the color package is imported in the '02x' macro files, you can check to
% see which driver is being used and tailor macros by driver by checking the
% value of '\Gin@driver{}', like so:
%
% \ifx\Gin@driver\pdftexdrivername
% <do something for pdftex driver>
% \else\ifx\Gin@driver\dvipsdrivername
% <do something for dvips driver>
% \fi
% \else
% ...
% \fi
\def\pdftexdrivername{pdftex.def}
\def\dvipsdrivername{dvips.def}