% Load 'color' package in a way that will work for both LaTeX and plain TeX files.
\input miniltx
\input color

% '@' needs to be processed in command names so we can access lower-level macros
% (e.g., '\current@color', '\Gin@driver') that will defined by the TeX engines
% to let us know what's doing the coloring, and to let us set colors.
\makeatletter

% Define names of color drivers
% You can check the name of the current driver using '\Gin@driver{}'
\def\pdftexdrivername{pdftex.def}
\def\dvipsdrivername{dvips.def}

% Command for setting the color for everything from here onward (even after this 
% this command appears in has terminated.)
\def\scholarsetcolor[#1]#2{%
{\csname color@#1\endcsname\current@color{#2}%
\ifx\Gin@driver\pdftexdrivername%
\pdfcolorstack0 push {\current@color}%
\else\ifx\Gin@driver\dvipsdrivername%
\special{color push \current@color}%
\else\typeout{Coloring not implemented for driver \Gin@driver}\INVALIDCOMMANDFORCEERROR%
\fi\fi
}}%
% Command for reverting a color set in an 'scholarsetcolor' command.
\def\scholarrevertcolor{%
\ifx\Gin@driver\pdftexdrivername%
\pdfcolorstack0 pop%
\else\ifx\Gin@driver\dvipsdrivername%
\special{color pop}%
\else\typeout{Coloring not implemented for driver \Gin@driver}\INVALIDCOMMANDFORCEERROR%
\fi\fi
}%

% Revert '@' top to just be a normal character.
\makeatother